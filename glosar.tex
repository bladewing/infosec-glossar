\documentclass{scrartcl}

\usepackage{fontspec} % For custom fonts
\usepackage{polyglossia} % For multilingual support
\setdefaultlanguage{german}

\usepackage{xltabular} % For tables that span multiple pages
\usepackage{booktabs} % For better looking tables

\title{Glossar Informationssicherheit und Datenschutz}
\author{Prof. Dr. Lukas Iffländer}
\date{Stand \today}

\begin{document}

\maketitle

\begin{xltabular}{\textwidth}{lX}
    \toprule
    Begriff & Definition \\
    \midrule
    \endhead
    \bottomrule
    \endfoot
    Angriffsfläche & Die Angriffsfläche (attack surface) ist die Gesamtheit aller Angriffsvektoren, die ein System angreifbar machen. \\
    Angriffsvektor & Ein Angriffsvektor (attack vector) ist eine Methode, mit der ein Angreifer ein System angreifen kann. \\
    Anonymität & Anonymität (anonymity) beschreibt die Eigenschaft, dass die Identität einer Person oder eines Systems nicht erkannt werden kann. \\
    Authentizität & Authentizität (authenticity) beschreibt die Eigenschaft, dass die Identität einer Person oder eines Systems verifiziert werden kann. \\
    CIA Triade & Die CIA Triade ist ein Modell, das die drei wichtigsten Ziele der Informationssicherheit beschreibt: Vertraulichkeit (Confidentiality), Integrität (Integrity) und Verfügbarkeit (Availability). \\
    DoS & Denial of Service. Ein Angriff, bei dem ein Server oder ein Infrastruktur in eine Zustand versetzt wird, in der ein Dienst nicht mehr zuverlässig angeboten werden kann. \\
    DDoS & Distributed Denial of Service. Ein Angriff, bei dem ein Server oder eine Infrastruktur durch eine Vielzahl von Anfragen überlastet wird. \\
    Integrität & Integrität (integrity) beschreibt die Eigenschaft, dass Informationen nicht unbemerkt verändert werden können. \\
    Sicherheitsmechanismus & Ein Sicherheitsmechanismus ist eine Methode oder Vorgehensweise zur Umsetzung einer Sicherheitsstrategie. \\
    Sicherheitsstrategie & Eine Sicherheitsstrategie (security policy) ist eine Formulierung der Aktionen, welche im System zulässig sind und welche nicht. \\
    Verbindlichkeit & Verbindlichkeit (non-repudiation) beschreibt die Eigenschaft, dass eine Kommunikation nicht abgestritten werden kann. \\
    Verfügbarkeit & Verfügbarkeit (availability) beschreibt die Eigenschaft, dass Informationen jederzeit verfügbar sind. \\
    Vertraulichkeit & Vertraulichkeit (confidentiality) beschreibt die Eigenschaft, dass Informationen nur von berechtigten Personen eingesehen werden können. \\
\end{xltabular}

\end{document}