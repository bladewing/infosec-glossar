\documentclass{scrartcl}

\usepackage{fontspec} % For custom fonts
\usepackage{polyglossia} % For multilingual support
\setdefaultlanguage{german}

\usepackage{xltabular} % For tables that span multiple pages
\usepackage{booktabs} % For better looking tables

\title{Glossar Informationssicherheit und Datenschutz}
\author{Prof. Dr. Lukas Iffländer}
\date{Stand \today}

\begin{document}

\maketitle

\begin{xltabular}{\textwidth}{lX}
    \toprule
    Begriff & Definition \\
    \midrule
    \endhead
    \bottomrule
    \endfoot
    Angriffsfläche & Die Angriffsfläche (attack surface) ist die Gesamtheit aller Angriffsvektoren, die ein System angreifbar machen. \\
    Angriffsvektor & Ein Angriffsvektor (attack vector) ist eine Methode, mit der ein Angreifer ein System angreifen kann. \\
    Anonymität & Anonymität (anonymity) beschreibt die Eigenschaft, dass die Identität einer Person oder eines Systems nicht erkannt werden kann. \\
    Asymmetrische Verschlüsselung & Asymmetrische Verschlüsselung ist eine Form der Verschlüsselung, bei der unterschiedliche Schlüssel zum Verschlüsseln und Entschlüsseln verwendet werden. \\
    Authentizität & Authentizität (authenticity) beschreibt die Eigenschaft, dass die Identität einer Person oder eines Systems verifiziert werden kann. \\
    Beschränkte Algorithmen & Beschränkte Algorithmen sind nicht publiziert und deren Sicherheit daher nicht neutral nachprüfbar. \\
    Blockbasierte Verschlüsselung & Blockbasierte Verschlüsselung ist eine Form der Verschlüsselung, bei der die Nachricht in Blöcke aufgeteilt wird, die dann verschlüsselt werden. \\
    Chiffretext & Chiffretext (cipher text) ist eine verschlüsselte Nachricht. \\
    Chosen-plaintext-Angriff & Ein Chosen-plaintext-Angriff ist ein Angriff, bei dem der Angreifer den Klartext bestimmter Nachrichten wählen kann und den Chiffretext dazu erhält. \\
    CIA Triade & Die CIA Triade ist ein Modell, das die drei wichtigsten Ziele der Informationssicherheit beschreibt: Vertraulichkeit (Confidentiality), Integrität (Integrity) und Verfügbarkeit (Availability). \\
    Cyphertext-Only-Angriff & Ein Cyphertext-Only-Angriff ist ein Angriff, bei dem der Angreifer nur den verschlüsselten Text kennt. \\
    DoS & Denial of Service. Ein Angriff, bei dem ein Server oder ein Infrastruktur in eine Zustand versetzt wird, in der ein Dienst nicht mehr zuverlässig angeboten werden kann. \\
    DDoS & Distributed Denial of Service. Ein Angriff, bei dem ein Server oder eine Infrastruktur durch eine Vielzahl von Anfragen überlastet wird. \\
    Integrität & Integrität (integrity) beschreibt die Eigenschaft, dass Informationen nicht unbemerkt verändert werden können. \\
    Klartext & Klartext (plaintext) ist eine unverschlüsselte Nachricht. \\
    Known-praintext-Angriff & Ein Known-plaintext-Angriff ist ein Angriff, bei dem der Angreifer sowohl den Klartext als auch den Chiffretext bestimmter Nachrichten kennt. \\
    Krypt(o)analyse & Kryptoanalyse ist die Wissenschaft des Brechens von kryptographisch verschlüsselten Nachrichten. \\
    Kryptographie & Kryptographie die Wissenschaft der sicheren Übertragung von Informationen. \\
    Kryptologie & Kryptologie ist die Wissenschaft der Kryptographie und Kryptoanalyse. \\
    One-Time-Pad & Ein One-Time-Pad ist eine Verschlüsselungsmethode, bei der der Schlüssel genauso lang wie die Nachricht ist und nur einmal verwendet wird. \\
    Sicherheitsmechanismus & Ein Sicherheitsmechanismus ist eine Methode oder Vorgehensweise zur Umsetzung einer Sicherheitsstrategie. \\
    Sicherheitsstrategie & Eine Sicherheitsstrategie (security policy) ist eine Formulierung der Aktionen, welche im System zulässig sind und welche nicht. \\
    Strombasierte Verschlüsselung & Strombasierte Verschlüsselung ist eine Form der Verschlüsselung, bei der die Nachricht Zeichen für Zeichen verschlüsselt wird. \\
    Symmetrische Verschlüsselung & Symmetrische Verschlüsselung ist eine Form der Verschlüsselung, bei der der gleiche Schlüssel zum Verschlüsseln und Entschlüsseln verwendet wird. \\
    Verbindlichkeit & Verbindlichkeit (non-repudiation) beschreibt die Eigenschaft, dass eine Kommunikation nicht abgestritten werden kann. \\
    Verfügbarkeit & Verfügbarkeit (availability) beschreibt die Eigenschaft, dass Informationen jederzeit verfügbar sind. \\
    Vertraulichkeit & Vertraulichkeit (confidentiality) beschreibt die Eigenschaft, dass Informationen nur von berechtigten Personen eingesehen werden können. \\
\end{xltabular}

\end{document}